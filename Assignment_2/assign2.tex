\let\negmedspace\undefined
\let\negthickspace\undefined
\documentclass[journal,12pt,twocolumn]{IEEEtran}
\usepackage{cite}
\usepackage{amsmath,amssymb,amsfonts,amsthm}
\usepackage{algorithmic}
\usepackage{graphicx}
\usepackage{textcomp}
\usepackage{xcolor}
\usepackage{txfonts}
\usepackage{listings}
%\usepackage{enumitem}
\usepackage{mathtools}
\usepackage{gensymb}
\usepackage[breaklinks=true]{hyperref}
\usepackage{tkz-euclide} % loads  TikZ and tkz-base
\usepackage{listings}
\usepackage[inline]{enumitem}
\DeclareMathOperator*{\Res}{Res}
\renewcommand\thesection{\arabic{section}}
\renewcommand\thesubsection{\thesection.\arabic{subsection}}
\renewcommand\thesubsubsection{\thesubsection.\arabic{subsubsection}}

\renewcommand\thesectiondis{\arabic{section}}
\renewcommand\thesubsectiondis{\thesectiondis.\arabic{subsection}}
\renewcommand\thesubsubsectiondis{\thesubsectiondis.\arabic{subsubsection}}

\def\inputGnumericTable{}                                 %%

\lstset{
frame=single, 
breaklines=true,
columns=fullflexible
}

\begin{document}

\newtheorem{theorem}{Theorem}[section]
\newtheorem{problem}{Problem}
\newtheorem{proposition}{Proposition}[section]
\newtheorem{lemma}{Lemma}[section]
\newtheorem{corollary}[theorem]{Corollary}
\newtheorem{example}{Example}[section]
\newtheorem{definition}[problem]{Definition}
\newcommand{\BEQA}{\begin{eqnarray}}
\newcommand{\EEQA}{\end{eqnarray}}
\newcommand{\define}{\stackrel{\triangle}{=}}

\bibliographystyle{IEEEtran}

\providecommand{\mbf}{\mathbf}
\providecommand{\pr}[1]{\ensuremath{\Pr\left(#1\right)}}
\providecommand{\qfunc}[1]{\ensuremath{Q\left(#1\right)}}
\providecommand{\sbrak}[1]{\ensuremath{{}\left[#1\right]}}
\providecommand{\lsbrak}[1]{\ensuremath{{}\left[#1\right.}}
\providecommand{\rsbrak}[1]{\ensuremath{{}\left.#1\right]}}
\providecommand{\brak}[1]{\ensuremath{\left(#1\right)}}
\providecommand{\lbrak}[1]{\ensuremath{\left(#1\right.}}
\providecommand{\rbrak}[1]{\ensuremath{\left.#1\right)}}
\providecommand{\cbrak}[1]{\ensuremath{\left\{#1\right\}}}
\providecommand{\lcbrak}[1]{\ensuremath{\left\{#1\right.}}
\providecommand{\rcbrak}[1]{\ensuremath{\left.#1\right\}}}
\theoremstyle{remark}
\newtheorem{rem}{Remark}
\newcommand{\sgn}{\mathop{\mathrm{sgn}}}

\newcommand{\solution}{\noindent \textbf{Solution: }}
\newcommand{\cosec}{\,\text{cosec}\,}
\providecommand{\dec}[2]{\ensuremath{\overset{#1}{\underset{#2}{\gtrless}}}}
\newcommand{\myvec}[1]{\ensuremath{\begin{pmatrix}#1\end{pmatrix}}}
\newcommand{\mydet}[1]{\ensuremath{\begin{vmatrix}#1\end{vmatrix}}}

\let\vec\mathbf


\vspace{3cm}

\title{
%	\logo{
   Assignment 2\\ \Large AI1110: Probability and Random Variables \\ \large Indian Institute of Technology Hyderabad
%	}
}
\author{ Aditya Garg \\ CS22BTECH11002 \\ 26 April 2023	
	
}	
% make the title area
\maketitle
\newpage
\bigskip
\renewcommand{\thefigure}{\theenumi}
\renewcommand{\thetable}{\theenumi}

\textbf{11.16.1.12}:
One urn contains two black balls (labelled B1 and B2) and one white ball. A
second urn contains one black ball and two white balls (labelled W1 and W2).
Suppose the following experiment is performed. One of the two urns is chosen
at random. Next a ball is randomly chosen from the urn. Then a second ball is
chosen at random from the same urn without replacing the first ball.

\begin{enumerate}[label=(\alph*)]
\item Write the sample space showing all possible outcomes

\item What is the probability that two black balls are chosen?

\item What is the probability that two balls of opposite colour are chosen?
\end{enumerate}


%Answer
\solution

Probability of an event $E$, written as $\pr{E}$
\begin{align}
P(E)=\displaystyle\frac{\text{Number of outcomes favourable to $E$}}{\text{Total Number of possible outcomes in sample space }}
\end{align}
Let the whit ball in first urn be 'W' and the black ball in second urn be 'B'.

\begin{enumerate}[label =(\alph*)]
\item Sample Space S:
\begin{align}
\cbrak{B_1B_2,B_2B_1,B_1W,WB_1,B_2W,WB_2,W_1W_2,W_2W_1,W_1B,BW_1,W_2B,BW_2}
\end{align} 
\begin{align}
    \therefore n(S) = 12
\end{align}

\item Let $E$ be event that 2 black balls are chosen,
The favourable outcomes are 
\cbrak{B_1B_2,B_2B_1}
\begin{align}
   \pr{E}&=\frac{2}{12}\\ 
        &=\frac{1}{6} \\
    \therefore \pr{E} &= \frac{1}{6}
\end{align}

\item Let $E$ be event that balls of opposite colours are chosen,
The favourable outcomes are 

\cbrak{B_1W,WB_1,B_2W,WB_2,W1_B,BW_1,W_2B,BW_2}
\begin{align}
    \pr{E}&=\frac{8}{12}\\
        &=\frac{2}{3} \\
    \therefore \pr{E} &= \frac{2}{3}
\end{align}
\end{enumerate}

\end{document}
