\let\negmedspace\undefined
\let\negthickspace\undefined
\documentclass[journal,12pt,twocolumn]{IEEEtran}
\usepackage{cite}
\usepackage{amsmath,amssymb,amsfonts,amsthm}
\usepackage{algorithmic}
\usepackage{graphicx}
\usepackage{textcomp}
\usepackage{xcolor}
\usepackage{txfonts}
\usepackage{listings}
%\usepackage{enumitem}
\usepackage{mathtools}
\usepackage{gensymb}
\usepackage[breaklinks=true]{hyperref}
\usepackage{tkz-euclide} % loads  TikZ and tkz-base
\usepackage{listings}
\usepackage[inline]{enumitem}
\DeclareMathOperator*{\Res}{Res}
\renewcommand\thesection{\arabic{section}}
\renewcommand\thesubsection{\thesection.\arabic{subsection}}
\renewcommand\thesubsubsection{\thesubsection.\arabic{subsubsection}}

\renewcommand\thesectiondis{\arabic{section}}
\renewcommand\thesubsectiondis{\thesectiondis.\arabic{subsection}}
\renewcommand\thesubsubsectiondis{\thesubsectiondis.\arabic{subsubsection}}

\def\inputGnumericTable{}                                 %%

\lstset{
frame=single, 
breaklines=true,
columns=fullflexible
}

\begin{document}

\newtheorem{theorem}{Theorem}[section]
\newtheorem{problem}{Problem}
\newtheorem{proposition}{Proposition}[section]
\newtheorem{lemma}{Lemma}[section]
\newtheorem{corollary}[theorem]{Corollary}
\newtheorem{example}{Example}[section]
\newtheorem{definition}[problem]{Definition}
\newcommand{\BEQA}{\begin{eqnarray}}
\newcommand{\EEQA}{\end{eqnarray}}
\newcommand{\define}{\stackrel{\triangle}{=}}

\bibliographystyle{IEEEtran}

\providecommand{\mbf}{\mathbf}
\providecommand{\pr}[1]{\ensuremath{\Pr\left(#1\right)}}
\providecommand{\qfunc}[1]{\ensuremath{Q\left(#1\right)}}
\providecommand{\sbrak}[1]{\ensuremath{{}\left[#1\right]}}
\providecommand{\lsbrak}[1]{\ensuremath{{}\left[#1\right.}}
\providecommand{\rsbrak}[1]{\ensuremath{{}\left.#1\right]}}
\providecommand{\brak}[1]{\ensuremath{\left(#1\right)}}
\providecommand{\lbrak}[1]{\ensuremath{\left(#1\right.}}
\providecommand{\rbrak}[1]{\ensuremath{\left.#1\right)}}
\providecommand{\cbrak}[1]{\ensuremath{\left\{#1\right\}}}
\providecommand{\lcbrak}[1]{\ensuremath{\left\{#1\right.}}
\providecommand{\rcbrak}[1]{\ensuremath{\left.#1\right\}}}
\theoremstyle{remark}
\newtheorem{rem}{Remark}
\newcommand{\sgn}{\mathop{\mathrm{sgn}}}

\newcommand{\solution}{\noindent \textbf{Solution: }}
\newcommand{\cosec}{\,\text{cosec}\,}
\providecommand{\dec}[2]{\ensuremath{\overset{#1}{\underset{#2}{\gtrless}}}}
\newcommand{\myvec}[1]{\ensuremath{\begin{pmatrix}#1\end{pmatrix}}}
\newcommand{\mydet}[1]{\ensuremath{\begin{vmatrix}#1\end{vmatrix}}}

\let\vec\mathbf


\vspace{3cm}

\title{
%	\logo{
   Assignment 1\\ \Large AI1110: Probability and Random Variables \\ \large Indian Institute of Technology Hyderabad
%	}
}
\author{ Aditya Garg \\ CS22BTECH11002 \\ 26 April 2023	
	
}	
% make the title area
\maketitle
\newpage
\bigskip
\renewcommand{\thefigure}{\theenumi}
\renewcommand{\thetable}{\theenumi}


\noindent  \textbf{10.15.1.4}:
 Which of the following cannot be the probability of an event ?
\vspace{7pt}


\begin{enumerate*}
   
\item \hspace{4pt}\(\displaystyle\frac{2}{3}\) \hspace{5pt} \item \hspace{4pt}\(-1.5\) \hspace{5pt} \item \hspace{4pt}\(15\%\) \hspace{5pt}  \item \hspace{4pt} \(0.7\) \hspace{5pt}     
\end{enumerate*}

\bigskip

%Answer
\noindent \textbf{Answer}:

Probability of an event \(E\) , written as \(P(E)\):

\medskip
\(P(E)=\displaystyle\frac{\text{Number of outcomes favourable to \(E\)}}{\text{Total Number of possible outcomes in sample space }}\)

\bigskip
From the definition of probability \(P(E)\) , number of favourable outcomes is always less than or equal to the number of all possible outcomes.


\begin{equation}
   \label{eq:prob}
   0 \le P(E)\le 1 
\end{equation}
\begin{enumerate}
    

\item \(\displaystyle\frac{2}{3}\)

\medskip
\indent Since \(0 \le \displaystyle\frac{2}{3} \le 1\) , From \eqref{eq:prob}

\medskip
\indent It can be probability of an event.

\bigskip
\item \(-1.5\)

\bigskip
\indent Since  \( -1.5 < 0 \) , From \eqref{eq:prob}

\medskip
\indent It cannot be a probability of any event.

\bigskip
\item \(15\%\)

\bigskip
\indent \(15\%=\displaystyle\frac{15}{100}\)

\bigskip
\indent Since \(0 \le \displaystyle\frac{15}{100} \le 1\) , From \eqref{eq:prob}

\medskip
\indent It can be probability of an event.

\bigskip
\bigskip

\item \(0.7\) \bigskip

Since \(0 \le 0.7 \le 1\), From \eqref{eq:prob} 
\medskip

It can be a probability of an event.
\end{enumerate}


\end{document}
